% !TeX root = ../main.tex

\chapter{论文格式规范}

\section{论文的文字及书写}

\subsection{论文的文字}

研究生学位论文一般用中文撰写,采用国家正式公布实施的简化汉字和法定的计量单位。也可以用英文撰写,但须同时提交用中文撰写的详细摘要。
\begin{enumerate}
  \item 来华留学生学位论文的目录、主体部分和致谢等可用英文撰写;但封面、独创性声明和权属声明应用中文撰写,硕士生须同时提交3000字左右的中文详细摘要,博士生须同时提交5000字左右的中文详细摘要。
  \item 外语专业的学位论文的目录、主体部分和致谢等应用所学专业相应的语言撰写;但封面、独创性声明和权属声明应用中文撰写,摘要应使用中文和所学专业相应的语言对照撰写。
\end{enumerate}

\subsection{论文的书写}

学位论文一律采用A4(70g)幅面白色纸张,封面、封底采用白色布纹纸张,中、英文扉页、独创性声明和使用授权书采用单面印刷,从中文摘要开始采用双面印刷。

\subsection{字体和字号}

\begin{itemize}
\item 章标题:三号黑体居中
\item 节标题:四号黑体居左
\item 条标题:小四号黑体居左
\item 主体部分:小四号宋体
\item 页码:五号宋体
\item 数字和字母: Times New Roman
\end{itemize}

\section{论文页面设置}

\subsection{页边距及行距}

学位论文的上边距:25mm;下边距:25mm;左边距:30mm;右边距:20mm。
章、节、条三级标题为单倍行距,段前、段后各设为0.5行(即前后各空0.5行)。
主体部分为1.5倍行距,段前、段后无空行(即空0行)。

\subsection{页眉}

页眉的上边距为15mm,页脚的下边距为15mm。页眉内容:页眉标注从论文主体部分开始(绪论或第一章),页眉用五号宋体,居中排列。奇偶页不同。奇数页页眉为章序及章标题,例如:“第四章  路基病害类型及分布规律”,偶数页页眉为“长安大学博士学位论文”或“长安大学硕士学位论文”。格式为页眉的文字内容之下划一条横线,线长与页面齐宽。

\subsection{页码}
论文页码从“主体部分”开始,直至“致谢”结束,用五号阿拉伯数字连续编码,页码位于页脚居中。

封面(中、英文扉页)、学位论文的独创性声明和权属声明不编入页码。

摘要、目录、图表清单、主要符号表用五号小罗马数字连续编码,页码位于页脚居中。

\section{名词术语}

科技名词术语及设备、元件的名称,应采用国家标准或部颁标准中规定的术语或名称。标准中未规定的术语要采用行业通用术语或名称。全文名词术语必须统一。特殊名词或新名词应在适当位置加以说明或注解。

采用英语缩写词时,除本行业广泛应用的通用缩写词外,文中第一次出现的缩写词应该用括号注明英文全称。

\section{物理量名称、符号与计量单位}

文中所用的物理量、符号与单位一律采用国家正式公布实施的《中华人民共和国法定计量单位》及国家标准《量和单位》(GB3100~3102)。

\section{图、表及其附注}

图和表应安排在主体部分中第1次提及该图、表的文字下方。当图或表不能安排在该页时,应安排在该页的下一页。

\subsection{图}

图包括曲线图、结构图、示意图、图解、框图、流程图、记录图、布置图、地图、照片、图版等。

图应具有“自明性”,即只看图、图题和图例,不阅读正文,就可理解图意。图的编号应采用阿拉伯数字分章依续编号,如:“图3.2”。

图题应明确简短,用五号宋体加粗,数字和字母为五号Times New Roman体加粗,图的编号与图题之间应空半角2格。图的编号与图题应置于图下方的居中位置。图内文字为5号宋体,数字和字母为5号Times New Roman体。曲线图的纵横坐标必须标注“量、标准规定符号、单位”,此三者只有在不必要标明(如无量刚等)的情况下方可省略。坐标上标注的量的符号和缩略词必须与正文中一致。

照片图要求主题和主要部分的轮廓鲜明,如用放大缩小的复制品,必须清晰,反差适中。照片上应有表示目的物尺寸的标度。

\subsection{表}

一律使用三线表,与文字齐宽,上下边线,线粗1.5 磅,表内线,线粗1 磅。例如表2-1;

表的编排,一般是内容和测试项目由左至右横读,数据依序竖读。表应有自明性。

表的编号应采用阿拉伯数字分章依续编号,如:“表2.5”。表题应明确简短,用五号宋体加粗,数字和字母为五号Times New Roman体加粗,表的编号与表题之间应空半角2格。表的编号与表题应置于表上方的居中位置。表内文字为5号宋体,数字和字母为5号Times New Roman体。

如某个表需要转页接排,在随后的各页上应重复表的编排。编号后跟表题(可省略)和“(续)”,如下所示:

表2.1  路基各边界热流密度(续)

续表应重复表头和关于单位的陈述。

\subsection{附注}

图、表中若有附注时,附注各项的序号一律用“附注 + 阿拉伯数字 + 冒号” ,如:“附注1:”。附注写在图、表的下方,一般采用5号宋体。

\section{公式}

文中公式的编号采用阿拉伯数字按章编排,用圆括号括起写在右边行末,其间不加虚线。如第一章第1个公式序号为“(1.1)”, 附录A中的第1个公式为“(A1)”等。文中引用公式时,一般用“见式(1.1)”或“由公式(1.1)”。

\section{注释}

学位论文中有个别名词或情况需要解释时,可加注说明,注释用页末注(将注文放在加注页的下端),而不用篇末注(将全部注文集中在文章末尾)和行中注(夹在论文主体部分中的注)。注号用阿拉伯数字上标标注,如:“注1”

\section{保密论文}

鼓励对学位论文进行去密处理,减少不必要的保密学位论文数量。去密处理时一般应去掉应用背景,与保密项目相关的技术指标和关键数据,使论文变成纯理论和技术的研究,达到可以在论文评审人员范围内公开或阅读的程度。对于技术和方法的保密,应该通过申请专利来保护,而不是把学位论文变为保密论文。

确实需要保密的论文由指导教师根据论文的情况提出并填写《长安大学涉密学位(毕业)论文定密审批表》,
校保密工作委员会按照国家规定的保密条例进行审批。
保密审批通过的论文需在封面直接把相应的“密级  % TODO: 打五角星
”及“保密期限”标注在右上角,密级按由低到高可分为“秘密”、“机密”、“绝密”三级。
